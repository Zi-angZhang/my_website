%        File: main.tex
%     Created: Tue Aug 18 03:00 PM 2020 +0
% Last Change: Tue Aug 18 03:00 PM 2020 +0
%
\documentclass[a4paper, 11pt]{article}
\usepackage[utf8]{inputenc}
\usepackage{tikz}
\usepackage[colorinlistoftodos, textsize=tiny]{todonotes}
\usepackage[super]{cite}
\bibliographystyle{ieeetr}
\usepackage[]{amsmath}
\numberwithin{equation}{subsection}
\usepackage{xcolor}
\usepackage[]{graphicx}
% color for general purposes
\definecolor{salmon}{RGB}{250, 128, 114}
\definecolor{grey}{RGB}{97, 97, 97}
\definecolor{bluegrey}{RGB}{96, 125, 139}
\definecolor{darkorange}{RGB}{245, 124, 0}
%color for listing block aka coding block
\definecolor{codegreen}{rgb}{0,0.6,0}
\definecolor{codegray}{rgb}{0.5,0.5,0.5}
\definecolor{codepurple}{rgb}{0.58,0,0.82}
\definecolor{codebackground}{rgb}{0.95,0.95,0.92}
\definecolor{codeterminalbackground}{rgb}{0.95,0.95,0.95}

\usepackage{listings}
\lstdefinestyle{python}{
	backgroundcolor=\color{codebackground},   
	commentstyle=\color{codegreen},
	keywordstyle=\color{magenta},
	numberstyle=\tiny\color{codegray},
	stringstyle=\color{codepurple},
	basicstyle=\ttfamily\footnotesize,
	breakatwhitespace=false,         
	breaklines=true,                 
	captionpos=b,                    
	keepspaces=true,                 
	numbers=left,                    
	numbersep=5pt,                  
	showspaces=false,                
	showstringspaces=false,
	showtabs=false,                  
	tabsize=4
}
\lstdefinestyle{terminal}{
	backgroundcolor=\color{codeterminalbackground},
	stringstyle=\color{black},
	basicstyle=\ttfamily\footnotesize,
	breakatwhitespace=false,
	breaklines=true,                 
	captionpos=b,                    
	keepspaces=true,                 
	showspaces=false,                
	showstringspaces=false,
	showtabs=false,                  
	tabsize=4,
	numbers=none
}
\lstset{language=Python, style=python}

\usepackage{hyperref}
\hypersetup{
	colorlinks=true, 
	linkcolor=grey, 
	filecolor=grey,      
	urlcolor=bluegrey,
}

\title{Material Science \& Artificial Intelligence}
\author{Ziang Zhang \\ \texttt{ziang.zhang@kaust.edu.sa}}
\date{\today}


\begin{document}
\maketitle
\tableofcontents

\section{Preface}

Artificial Intelligence (AI) is growing rapidly as the computer industry emerging. Tons of deep learning algorithms are developed to solve daily issues but few, compared to hot fields, i.e. Computer Vision. The initial purpose of this project is to review the existing methods of machine learning working with material science. The origin of this field can date back to long long ago, years before deep learning fever.

\section{available packages}

\subsection{USPEX}

 The full name of \textbf{USPEX} is ``Universal Structure Predictor: Evolutionary Xtallography''. Official website: https://uspex-team.org. 

 USPEX code solves ``chemical composition to crystalline solids'' problems.  In addition to crystal structure prediction, USPEX can work in other dimensionalities and predict the structure of nanoparticles, polymers, surfaces, interfaces and 2D-crystals. It can predict stable chemical compositions and corresponding crystal structures, given just the names of the chemical elements.

\newpage
\bibliography{references}
\end{document}

