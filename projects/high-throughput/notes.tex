%        File: notes.tex
%     Created: Wed Sep 09 10:00 AM 2020 +0
% Last Change: Wed Sep 09 10:00 AM 2020 +0
%
\documentclass[a4paper, 11pt]{article}
\usepackage[utf8]{inputenc}
\usepackage{tikz}
\usepackage[colorinlistoftodos, textsize=tiny]{todonotes}
\usepackage[super]{cite}
\bibliographystyle{ieeetr}
\usepackage[]{amsmath}
\numberwithin{equation}{subsection}
\usepackage{xcolor}
\usepackage[]{graphicx}
% color for general purposes
\definecolor{salmon}{RGB}{250, 128, 114}
\definecolor{grey}{RGB}{97, 97, 97}
\definecolor{bluegrey}{RGB}{96, 125, 139}
\definecolor{darkorange}{RGB}{245, 124, 0}
%color for listing block aka coding block
\definecolor{codegreen}{rgb}{0,0.6,0}
\definecolor{codegray}{rgb}{0.5,0.5,0.5}
\definecolor{codepurple}{rgb}{0.58,0,0.82}
\definecolor{codebackground}{rgb}{0.95,0.95,0.92}
\definecolor{codeterminalbackground}{rgb}{0.95,0.95,0.95}

\usepackage{listings}
\lstdefinestyle{python}{
	backgroundcolor=\color{codebackground},   
	commentstyle=\color{codegreen},
	keywordstyle=\color{magenta},
	numberstyle=\tiny\color{codegray},
	stringstyle=\color{codepurple},
	basicstyle=\ttfamily\footnotesize,
	breakatwhitespace=false,         
	breaklines=true,                 
	captionpos=b,                    
	keepspaces=true,                 
	numbers=left,                    
	numbersep=5pt,                  
	showspaces=false,                
	showstringspaces=false,
	showtabs=false,                  
	tabsize=4
}
\lstdefinestyle{terminal}{
	backgroundcolor=\color{codeterminalbackground},
	stringstyle=\color{black},
	basicstyle=\ttfamily\footnotesize,
	breakatwhitespace=false,
	breaklines=true,                 
	captionpos=b,                    
	keepspaces=true,                 
	showspaces=false,                
	showstringspaces=false,
	showtabs=false,                  
	tabsize=4,
	numbers=none
}
\lstset{language=Python, style=python}

\usepackage{hyperref}
\hypersetup{
	colorlinks=true, 
	linkcolor=grey, 
	filecolor=grey,      
	urlcolor=bluegrey,
}

\title{high throughput}
\author{Ziang Zhang \\ \texttt{ziang.zhang@kaust.edu.sa}}
\date{\today}


\begin{document}
\maketitle
\tableofcontents

How to find papers: Structure and Mechanism > Computational Materials Science > Electronic Structure Theory > Density Functioal Theory.

\section{pipeline of high throughput research}

The discovery design of different mateials are summarized

\subsection{2D materials}

\subsubsection{Define the object} First, the definition of layered buld and 2D materials should be stated before starting the research. In the reference paper \cite{zhang_high-throughput_2019-1}, layered bulks are defined as the materials consisting of periodic layered structure connected by \textbf{noncovalent} bonds, such as hydrogen bonds, London dispersion interactions, and interstitial cations. The definition of 2D materials is given as the materials with a finite thickness of a few atoms in one dimension and infinite extension in the other two dimensions. In reality, the thickness of 2D materials could achieve a nanometer and the length of the other two dimensions should be large enough.

\subsubsection{Find resources} The mateirals database was listed:

\begin{itemize}
  \item ICSD
  \item Crystallography Open Database (COD)
  \item The Pauling File
  \item Materials Project databse [mateials API]
  \item AFLOWLIB consortium
  \item Open Quantum Materials Database
\end{itemize}

And other databases containing the properties of materials:

\begin{itemize}
  \item MaterialsWeb online database
  \item computational materials repository
  \item Midwest Nano Infrastructure Corridor 2D database
  \item Materials Cloud platform
\end{itemize}

In addition, effective and rapid large-scale data analysis is important in high-throughput computations. \textbf{Python Materials Genomics (Pymatgen)} integrating with Materials API and COD, is one of the typical tools for materials analysis.

\subsection{Theoretical methods}

\subsubsection{DFT} For DFT frameworks, the choice for exchange-correlation functional has great influence on computational results.

\begin{itemize}
  \item Local density approximation(LDA), KS DFT
  \item Generalized gradient approximation (CGA)
\end{itemize}

For layered bulk materials, the \textbf{van der Walls force }for weak interactions should be considered. Parameterized density functionals (DFs), DFT-D, nonlocal vdW-DFs, effective one-electron potentials, named disperion-correcting potentials (DCPs). Moreover, the random-phase approximation was also widly applied in the computaion of vdW interactions and empirically reported to be most accurate among the commonly used methods. 

There are many computational programs to execute DFT calculation, including VASP, CASTEP, Quantum Espresso, and ABINIT. \textbf{Generally, the 2D layers are simulated in 3D cell and to calculate them under periodic boundary conditions; a large space should be inserted between the repeated sheets}??\footnote{? periodic conditions on two dimension?}.

To apply DFT in strongly correlated systems, GW (Green's function and screened Coulomb interaction) was proposed to predict an accurate band gap of semiconductors and predict the excited-stated properties. Besides, hybrid DFT which would consume less computing time than GW method. Typically, \textit{Heyd-Scuseria-Ernzerhof(HSE)} hybrid functional which includes 25\% short range Hartree Fock exhcange, sould improve the accuracy of band structure. Moreover, the DFT + U treatment, adding Hubbard interaction U to the stardard DFT functional, provides a correct treatment for the strongly correlated materials.

\subsection{screening pipeline}

2012, Bjorkman scaned the ICSD dataset by applying geometric criteria.

\begin{enumerate}
  \item find layered compounds that the layers are perpendicular to the crystallographic $c$ axis.
  \item find compunds with high symmetry\footnote{for computational reasons?}
  \item find compunds which has high packing ratios
  \item find compounds with large gaps along the $c$ axis
  \item output the sctructures with a gap that the distance between the neighboring atoms across the gap is larger than the sum of their covalent radii.
\end{enumerate}

Aston et al. proposed topoloty-scaling algorithm to screen layered compounds in MP database. The algorithm can identify additional layered materials with the layers along other axes, materials with thick layers that the packing ratio is higher than normally layered materials, and materials with gaps along multiple axes.


\section{Computational screening of electrolyte materials}













\section{Appendix}

\subsection{Dirac cone}

\subsection{conductivity}

\subsection{hexagonal boron nitride h-BN}

\subsection{Transition metal dichalcogenide}

\subsection{MXenes}

\subsection{phosphorene}

\subsection{field-effect transistor}

\subsection{photothermal conversion}

\subsection{energy storage devices}

\subsection{surfactant addition}

\subsection{reaggregation}


\newpage
\bibliography{references}
\end{document}

